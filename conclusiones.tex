Para concluir, quisiera mencionar que encuentro muchas similitudes entre las opiniones de Bayer, Sábato, García y el desarrollismo,
probablemente por ser pensamientos de una misma corriente (aunque el desarrollismo se inició por el 1930).

Quiero hacer notar que, de todos los autores, encuentro en Sábato la mayor riqueza para analizar en cuanto a políticas tecnológicas, ya que se dedico a ese tema en particular con mucho énfasis. Si bien Bayer se enfoca principalmente en el tema de la autonomía nacional, el análisis que hace de las políticas tecnológicas es similar al que realiza Sábato, y seguramente ambos coincidirían en muchos puntos, principalmente en el rol y responsabilidades del Estado a la hora de conquistar la autonomía.

En cuanto a García, creo que estaría de acuerdo con Bayer y Sábato en materia de autonomía nacional, pero me queda por analizar más a fondo cual era su postura en cuanto a política tecnológica.

Encuentro en los 3 autores mencionados muchas de las ideas del desarrollismo (una participación activa del Estado, necesidad de industrialización, etc.), y creo que podrían considerarse desarrollistas, hasta cierto punto.
Sin embargo, analizando más a fondo las ideas centrales del desarrollismo, pareciera ser que el enfoque está puesto plenamente en buscar políticas de estado que permitan un mayor crecimiento a los países en vías de desarrollo, contrarrestando la caída de los términos de intercambio a largo plazo, esperando que dicho crecimiento se \textit{derrame} de alguna forma en toda la sociedad. Creo que los demás autores analizados estarían fuertemente en contra de esto, marcando que hacen falta políticas claras de distribución de la riqueza y bienestar social, para que el crecimiento logrado no caiga en unas pocas manos.

Por último, no encontré diferencias fuertes en las ideas principales de los diferentes autores (Bayer, Sábato y García), sino diferencias de enfoques.
