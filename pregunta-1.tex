\paragraph{Enunciado}
El artículo planea la relación entre la autonomía tecnológica y la autonomía
general de una nación (como oposición a su dependencia).
Comparar la posición de Bayer con la Jorge Sábato y la de Rolando García con relación al mismo tema.

\subsection*{Repaso de la posición de Bayer}

Gustavo F. Bayer define la \textbf{autonomía general de una nación}  como la capacidad
de un estado nacional para actuar según sus propios intereses. Dicha capacidad
responde a un proceso dinámico en el que no basta con simplemente conquistar la autonomía,
sino que es necesario mantenerla una vez obtenida.

Luego, caracteriza las bases de la autonomía nacional:
\begin{itemize}
    \item Una configuración de poder nacional, que ocurre cuando la acción del estado nacional
        influencia nitidamente las acciones de los otros.
    \item Autosuficiencia nacional, que permite minimizar la capacidad de influencia
        de los otros estados sobre el nuestro.
\end{itemize}

En este escenario, Bayer plantea que el rol de la política tecnólogica es el de:
\begin{itemize}
    \item Suministrar a corto plazo elementos útiles para la ampliación del poder nacional.
        Por ejemplo: explotación y transformación de materias primas. Es decir,
        deberá tender a crear condiciones para la explotación del excedente potencial de los recursos
        naturales. \textbf{Debe lograr adaptar las tecnologías transferidas del exterior a los
        recursos nacionales}.
    \item Satisfacer las necesidades sociales, buscando un aumento de la calidad de vida
        en la población y no solo el aumento de la producción como meta principal.
        Se trata de crear condiciones de oferta para satisfacer necesidades ya existentes,
        y no de introducir tecnología que precisa un mercado consumidor,
        generalmente limitado a las capas de mayor poder adquisitivo.
\end{itemize}

En resumen, la política tecnológica deberá permitir conquistar la autonomía a corto
plazo, explotando las potencialidades existentes y desarrollandose respetando el siguiente orden:
\begin{itemize}
    \item Recursos nacionales disponibles.
    \item Necesidad de utilización interna.
    \item Posibilidades de su utilización externa.
\end{itemize}

\subsection*{Comparación con la posición de Jorge Sábato}

\subsection*{Comparación con la posición de Rolando García}
