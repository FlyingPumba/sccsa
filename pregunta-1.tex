\paragraph{Enunciado}
El artículo planea la relación entre la autonomía tecnológica y la autonomía
general de una nación (como oposición a su dependencia).
Comparar la posición de Bayer con la Jorge Sábato y la de Rolando García con relación al mismo tema.

\subsection*{Repaso de la posición de Bayer}

Gustavo F. Bayer define la \textbf{autonomía general de una nación}  como la capacidad
de un estado nacional para actuar según sus propios intereses. Dicha capacidad
responde a un proceso dinámico en el que no basta con simplemente conquistar la autonomía,
sino que es necesario mantenerla una vez obtenida.

Luego, caracteriza las bases de la autonomía nacional:
\begin{itemize}
    \item Una configuración de poder nacional, que ocurre cuando la acción del estado nacional
        influencia nitidamente las acciones de los otros.
    \item Autosuficiencia nacional, que permite minimizar la capacidad de influencia
        de los otros estados sobre el nuestro.
\end{itemize}

En este escenario, Bayer plantea que el rol de la política tecnólogica es el de:
\begin{itemize}
    \item Suministrar a corto plazo elementos útiles para la ampliación del poder nacional.
        Por ejemplo: explotación y transformación de materias primas. Es decir,
        deberá tender a crear condiciones para la explotación del excedente potencial de los recursos
        naturales. \textbf{Debe lograr adaptar las tecnologías transferidas del exterior a los
        recursos nacionales}.
    \item Satisfacer las necesidades sociales, buscando un aumento de la calidad de vida
        en la población y no solo el aumento de la producción como meta principal.
        Se trata de crear condiciones de oferta para satisfacer necesidades ya existentes,
        y no de introducir tecnología que precisa un mercado consumidor,
        generalmente limitado a las capas de mayor poder adquisitivo.
\end{itemize}

En resumen, la política tecnológica deberá permitir conquistar la autonomía a corto
plazo, explotando las potencialidades existentes y desarrollandose respetando el siguiente orden:
\begin{itemize}
    \item Recursos nacionales disponibles.
    \item Necesidad de utilización interna.
    \item Posibilidades de su utilización externa.
\end{itemize}

\subsection*{Comparación con la posición de Jorge Sábato}

Notas Sábato:
- no al colonialismo mental
- pensar el desarrollo tecnologico como motor del desarrollo social y economico del país.
- solo mediante ese manejo autonomo tecnologico podra una nacion comenzar a marchar
    en la dirección que eventualmente le permitira disponer en cada caso de la tecnologia más ajustada
    a sus propios objetivos, más respestuosa de su acervo cultural, más conveniente para sus propias
    necesidades y más adecuadas a sus dotaciones de recursos y factores.

- tecnologia nacional: manejas la tencologia como mas le convenga al país,
- ¿Qué desarrollar y que comprar ? A veces conviene desarrollar primero, para luego importar en mejores condiciones.
- La relacion entre politica tecnologica y economica, es determinante.
- En particular, el uso del poder de compra del estado es el instrumento más importante

- se va a alcanzar autonomia tecnologica: no como autonomia que reniega de la participacion de tecnologica extranjera,
    sino como capacidad tecnologica autonoma, de generacion de tecnologia y de incorporacion de tecnologia estranjera a modo de configuracion de paquetes.
    Lo contrario sería la desagregacion de paquetes tecnologicos, es decir: una participacion de componentes nacionales.

- dependencia tecnologica: incorporación sumisa de las tecnologías disponibles por razones de disponibilidades y precios en el mercado.
- autonomia: capacidad de desarrollo local: construir el triangulo, interrelacinando los tres vertices. Solo puede pasar si el estado tiene
    una participacion clave en la estructura productiva.
- presidente de SERBA: Empieza identificando cuales son los productos en los cuales puede haber desarrollo
    tecnologico local y en los cuales podíamos prescindir de componentes importados.
- Industrialización sustitutiva de importaciones.
- Regimenes de promocion del desarrollo industrial, bastante protectivos: proteger a la industria naciente.
- De vuelta: el compre nacional es un factor clave para la autonomia tecnologica, muestra una preferencia en cuanto proveedores externos.
- Regulación de transferencia de tecnología extranjera

-Por tanto, la desagregación tecnológica es también un instrumento de política industrial que busca maximizar
la participación propia en la ejecución de proyectos de inversión por medio del incremento del componente tecnológico
 nacional en la producción de un bien o de un servicio. Para eso es preciso investigar, en detalle, los agregados humanos,
  económicos y técnicos de cada uno de los proyectos de inversión.

% http://www.fundacionsadosky.org.ar/seminario-ciencia-y-tecnologia-en-el-pensamiento-de-jorge-sabato-oscar-varsavsky-y-amilcar-herrera/

\subsection*{Comparación con la posición de Rolando García}
