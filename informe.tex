\documentclass[a4paper]{article}
\usepackage[spanish]{babel}
\usepackage[utf8]{inputenc}
\usepackage{fancyhdr}
\usepackage{charter} % tipografia
%\usepackage{graphicx}
\usepackage[pdftex]{graphicx}
\usepackage{sidecap}
\usepackage{caption}
\usepackage{subcaption}
\usepackage{booktabs}
\usepackage{makeidx}
\usepackage{float}
\usepackage{amsmath, amsthm, amssymb}
\newtheorem{theorem}{Teorema}[section]
\newtheorem{corollary}{Corolario}[theorem]
\newtheorem{lemma}[theorem]{Lema}
\usepackage{amsfonts}
\usepackage{sectsty}
\usepackage{charter}
\usepackage{wrapfig}
\usepackage{listings}
\usepackage{hyperref} % links
\usepackage{algorithm} %http://www.ctan.org/pkg/algorithms
\usepackage{algorithmic}
\input{codesnippet}
\input{page.layout}
% \setcounter{secnumdepth}{2}
\usepackage{underscore}
\usepackage{caratula}
\usepackage{url}
\usepackage[usenames,dvipsnames]{xcolor}
\lstset{
    language=C++,
    basicstyle=\ttfamily,
    keywordstyle=\color{blue}\ttfamily,
    stringstyle=\color{red}\ttfamily,
    commentstyle=\color{ForestGreen}\ttfamily,
    morecomment=[l][\color{magenta}]{\#}
}

% *********************** %
% Ejercicio 2 - Agregado
% Grafos
\usepackage{tikz}
\usetikzlibrary{positioning}
\usetikzlibrary{graphs}
\usetikzlibrary{calc}
\usetikzlibrary{arrows}
% Otros
\usepackage{arrayjobx}
\usepackage{natbib}
\usepackage{enumitem}
\usepackage{multicol}
\usepackage{pgfplots}
% *********************** %

\begin{document}
\thispagestyle{empty}
\materia{Seminario sobre Computación, Ciencia y Sociedad en Argentina}
\submateria{Primer Cuatrimestre de 2016}
\titulo{Autonomía nacional y política científica\\ y tecnológica}
\subtitulo{Gustavo F. Bayer}
\integrante{Iv\'an Arcuschin}{678/13}{iarcuschin@gmail.com}
\maketitle
% no footer on the first page
%\thispagestyle{empty}
%\newpage
%\tableofcontents

\newpage
\section*{Repaso de la posición de Gustavo Bayer}
Gustavo Bayer define la \textbf{autonomía general de una nación}  como la capacidad
de un estado nacional para actuar según sus propios intereses. Dicha capacidad
responde a un proceso dinámico en el que no basta con simplemente conquistar la autonomía,
sino que es necesario mantenerla una vez obtenida.

Luego, caracteriza las bases de la autonomía nacional:
\begin{itemize}
    \item Una \textbf{configuración de poder nacional}, que ocurre cuando la acción del estado nacional
        influencia nitidamente las acciones de los otros.
    \item \textbf{Autosuficiencia nacional}, que permite minimizar la capacidad de influencia
        de los otros estados sobre el nuestro.
\end{itemize}

En este escenario, Bayer plantea que el rol de la \textbf{política tecnólogica} es el de:
\begin{itemize}
    \item Suministrar a corto plazo elementos útiles para la ampliación del poder nacional.
        Por ejemplo: explotación y transformación de materias primas. Es decir,
        deberá tender a crear condiciones para la explotación del excedente potencial de los recursos
        naturales. \textbf{Debe lograr adaptar las tecnologías transferidas del exterior a los
        recursos nacionales}.
    \item \textbf{Satisfacer las necesidades sociales}, buscando un aumento de la calidad de vida
        en la población y no solo el aumento de la producción como meta principal.
        Se trata de crear condiciones de oferta para satisfacer necesidades ya existentes,
        y no de introducir tecnología que precisa un mercado consumidor,
        generalmente limitado a las capas de mayor poder adquisitivo.
\end{itemize}

Y el rol de la política cientifica:
\begin{itemize}
    \item Suministrar a largo plazo, los conocimientos necesarios para la sustentación de la autonomía. En particular, \textbf{ampliar y profundizar los conocimientos}, dirigidos por las necesidades del sistema en si, y no caer en el academicismo o la torre de marfil.
        Además se debe buscar una fuerte integración entre las diversas areas del conocimiento adquirido.
    \item Buscar la \textbf{autosuficiencia del sistema cientifico}, buscando la autosustentación, que luego le permitirá actuar como factor de activación de las potencialidades de autosuficiencia nacional a través de la calificación.
\end{itemize}

En resumen, la política tecnológica deberá permitir conquistar la autonomía a corto
plazo, explotando las potencialidades existentes y desarrollandose respetando el siguiente orden:
\begin{itemize}
    \item Recursos nacionales disponibles.
    \item Necesidad de utilización interna.
    \item Posibilidades de su utilización externa.
\end{itemize}

Y luego la política cientifica será la encargada de buscar nuevas potencialidades a ser explotadas, con el fin de mantener la autonomía alcanzada. Además, deberá evitar el clientelismo cientifico, donde el cientifico justifica su investigación diciendo que sirve para el sistema, pero en realidad no.

\vspace{0.5em}

Para finalizar, Bayer hace incapie en que \textbf{debe existir una decisión política de conquista de la autonomía nacional}. No se puede esperar la conquista de la autonomía a partir del crecimiento vegetativo de sus bases de poder y/o autosuficiencia, ya que si una nación que tiene potencialidades de poder no ha logrado su autonomía, es porque existen patrones de dependencia que le impiden despegarse del yugo de los paises desarrollados, junto con barreras sociales o materiales que le impiden alcanzar la autosuficiencia.

En caso de que está decision política no existiese, cualquier crecimiento económico o desarrollo deberá ser encarado con reservas: el crecimiento económico, como benefactor de determinadas capas sociales, concentrador; el crecimiento cientficio, desvinculado de la sociedad como un todo, un crecimiento por el crecimiento (la llamada \textit{torre de marfil}).


\newpage
\section*{Pregunta 1}
\paragraph{Enunciado}
El artículo planea la relación entre la autonomía tecnológica y la autonomía
general de una nación (como oposición a su dependencia).
Comparar la posición de Bayer con la Jorge Sábato y la de Rolando García con relación al mismo tema.

\subsection*{Comparación con la posición de Jorge Sábato}

Abarcaré la posición de Jorge A. Sábato principalmente desde su artículo más conocido con Natalio Botana: \textit{La ciencia y
la tecnología en el desarrollo futuro de América Latina}.

En dicho artículo, los autores presentan un triangulo de relaciones (como el detallado en la Figura \ref{triangulo}) en el cual cada vertice
corresponde a un actor:
\begin{itemize}
    \item Gobierno: el conjunto de roles institucionales que tienen como objetivo formular políticas y movilizar recursos de y hacia los otros vértices a través de los procesos legislativo y administrativo.
    \item Infraestructura cientifico-tecnológica: el conjunto de sistemas educativos, institutos de investigación, academias de ciencias, etc.
    \item Estructura productiva: el conjunto de sectores productivos que provee los bienes y servicios que demanda una determinada sociedad.
\end{itemize}

\begin{figure}[H]
    \centering
    \includegraphics[width=0.6\textwidth]{imagenes/sabato.png}
    \caption{Triangulo de Sábato/Botana}\label{triangulo}
\end{figure}

% Luego, cada vertice tiene responsabilidades y se espera que se relacione con los demás de ciertas maneras.

En el artículo mencionado, Sábato habla de autonomía como la capacidad de decisión propia, y menciona que esta es el resultado de un proceso deliberado de interrelaciones entre los vértices.

Entonces, las posturas de Sábato en cuanto a la relación entre autonomía tecnológica y
autonomía nacional se pueden analizar (¿quizás parcialmente?) desde las responsabilidades que plantea para los distintos vertices, junto con sus relaciones.
% Este proceso se establece a través del flujo de demandas que circulan en sentido vertical (gobierno - los otros dos vertices) y en sentido horizantal.

Sábato distingue tres tipos de relaciones posibles:

\begin{itemize}
    \item \textit{Intrarelaciones}: las que suceden entre los diferentes componentes de un mismo vértice.
    \item \textit{Interrelaciones}: las que suceden entre dos vertices de un mismo triangulo.
    \item \textit{Extrarelaciones}: las que suceden entre dos vertices de triangulos distintos.
\end{itemize}

De las \textbf{intrarelaciones}, basta mencionar que deben estructurarse con vista a garantizar una determinada \textit{capacidad} para generar, incorporar o transformar demandas en un producto final que es la innovacion cientifico-tecnologica. En particular:

% Si hablamos de relaciones internas dentro de cada vertice, \textit{intrarelaciones}, éstas tienen por objetivo transformar a estos centros de convergencia
% en centros capaces de generar, incorporar y transformar demandas en un producto final que es la innovacion cientifico-tecnologica. Es decir:

\begin{enumerate}
    \item El vertice-gobierno requiere la capacidad para realizar una \textit{acción deliberada} en el ambito de políticas cientifico-tecnologicas, con el fin de formular un cuerpo de doctrina, de principios y estrategias capaces de fijar metas posibles, cuyo logro depende de una serie de decisiones politicas, de la asignación de recursos y de la programación cientifico-tecnologica.
    \item El vertice-infraestructura cientifico-tecnologica requiere la \textit{capacidad creadora}: un cientifico mediocre producirá ideas mediocres, por más dinero que se les inyecte.
    \item El vertice-estructura productiva requiere \textit{capacidad empresarial}, publica o privada, que si las definiremos según las ideas desarrolladas por Schumpeter, como aquella función que ``consiste en reformar o revolucionar el sistema de produccion, explotando un invento o, de manera más general, una posibilidad técnica no experimentada para producir una mercancía nueva o una mercancía antigua por un método nuevo, para abir una nueva fuente de previsión de materias primas o una nueva salida para los productos, para reorganizar una industria, etc''.
\end{enumerate}

ALGUNA CONCLUSION MÁS ?
Este segundo item está fuertemente relacionado con las ideas de Bayer de que las politicas tecnologicas deben buscar explotar las potencialidades existentes.

\vspace{0.5em}

En cuanto a las \textbf{interrelaciones}, Sábato les asigna una responsabilidad mayuscula: la generación de una capacidad de decisión propia en el campo cientifico-tecnológico es el \textit{resultado de un proceso deliberado de interrelaciones} entre el vertice-gobierno, el vertice-infraestructura cientifico-tecnologica y el vertice-estructura productiva.

Además, Sábato señala que el vértice de la infraestructura cientifico-tecnologica depende vitalmente de la acción deliberada del gobierno, entendido en un sentido muy amplio, sobre todo en lo que se refiere a asignación de recursos. Sin embargo, el vertice-gobierno juega tambien el papel de centro impulsor de demandas hacia la infraestructura cientifico-tecnologica, y es aquí donde reside la dificultad mayor en el modo como se concebirá la formulación de programas una vez tomada la decisión política.

\vspace{0.5em}

A continuación Sábato pasa a analizar las \textbf{extrarelaciones}.

En una sociedad donde funciona el triángulo de relaciones, las aperturas que se realicen hacia el exterior en materia de exportación de ciencia y tecnología original o de adaptación de tecnología importada, producen beneficios reales, ya sea a corto o largo plazo.

Muy distinto es cuando estas aperturas se realizan entre vertices aislados con un triangulo plenamente desarrollado. Este problema es muy caracteristico de las sociedades latinoamericanas, y explica un sin fin de prolemas, como por ejemplo el éxodo de talentos. Mientras en nuestras sociedades el cientifico se encuentra desvinculado y aislado frente al gobierno y a la estructura productiva, en el nuevo lugar de trabajo, al cual lo conduce su exilio cultural, está automaticamente amparado por instituciones o centros de investigación que, a su vez, se encuentran insertos en el sistema de relaciones planteado.

\vspace{0.5em}

Habíendo dicho esto, Sábato afirma que en América Latina no existe un sistema de relaciones como el mencionado, ni tampoco hay conciencia acerca de la necesidad impostergable de establecerlo, por lo que hay una doble exigencia para las naciones en vías de desarrollo que busquen alcanzar su autonomía:
\begin{itemize}
    \item Crear una conciencia global para que nuestras sociedades asuman este problema en sus dimensiones reales.
    \item Actuar eficazmente sobre aquellos sectores en los cuales se podrían optimizar los recursos escasos en función del sistema de relaciones perseguido.
\end{itemize}

Además, Sabáto reafirma: corresponde al sector gubernamental formular una política tendiente a \textit{acoplar} la infraestructura cientifico-tecnologica al proceso de producción, ya sea creando los centros que así lo permitan o relacionando los centros ya existentes.

\vspace{0.5em}

Por último, quisiera dar un ejemplo concreto que presenta Sábato en su artículo \textit{Empresas y fábricas de tecnología}, de la creación en 1971 de una empresa Argentina estatal que responde a necesidades de desarrollo existentes en ese momento, la \textit{Empresa Nacional de Investigación y Desarrollo Eléctrico S.A.} (\texttt{ENIDE}). Su objetivo fundamental estaba definido en su estatuto como ``Producir, distribuir, comprar, vender, exportar, importar e intercambiar conocimiento técnico-científico en el campo de la energía eléctrica y sus aplicaciones'', y su creación obedecía a diversas circunstancias:

\begin{itemize}
    \item La existencia de un mercado importante y en rápido crecimiento.
    \item En el campo de la energía eléctrica, la Argentina era neto importador de tecnología.
    \item Existía capacidad científico-tecnológica apta para la producción de tecnología eléctica. La demanda interna era, sin embargo, muy escasa y de poca significación cualitativa.
    \item Por tratarse de un tipo de actividad que tenía poca tradición en el país, sobre cuya necesidad no existía aún conciencia clara y que requería capital de riesgo, sólo el Estado estaba en condiciones de ponerla en marcha.
\end{itemize}

% Notas Sábato:
% - no al colonialismo mental
% - pensar el desarrollo tecnologico como motor del desarrollo social y economico del país.
% - solo mediante ese manejo autonomo tecnologico podra una nacion comenzar a marchar
%     en la dirección que eventualmente le permitira disponer en cada caso de la tecnologia más ajustada
%     a sus propios objetivos, más respestuosa de su acervo cultural, más conveniente para sus propias
%     necesidades y más adecuadas a sus dotaciones de recursos y factores.
%
% - tecnologia nacional: manejar la tencologia como mas le convenga al país,
% - ¿Qué desarrollar y que comprar ? A veces conviene desarrollar primero, para luego importar en mejores condiciones.
% - La relacion entre politica tecnologica y economica, es determinante.
% - En particular, el uso del poder de compra del estado es el instrumento más importante
%
% - se va a alcanzar autonomia tecnologica: no como autonomia que reniega de la participacion de tecnologica extranjera,
%     sino como capacidad tecnologica autonoma, de generacion de tecnologia y de incorporacion de tecnologia estranjera a modo de configuracion de paquetes.
%     Lo contrario sería la desagregacion de paquetes tecnologicos, es decir: una participacion de componentes nacionales.
%
% - dependencia tecnologica: incorporación sumisa de las tecnologías disponibles por razones de disponibilidades y precios en el mercado.
% - autonomia: capacidad de desarrollo local: construir el triangulo, interrelacinando los tres vertices. Solo puede pasar si el estado tiene
%     una participacion clave en la estructura productiva.
% - presidente de SERBA: Empieza identificando cuales son los productos en los cuales puede haber desarrollo
%     tecnologico local y en los cuales podíamos prescindir de componentes importados.
% - Industrialización sustitutiva de importaciones.
% - Regimenes de promocion del desarrollo industrial, bastante protectivos: proteger a la industria naciente.
% - De vuelta: el compre nacional es un factor clave para la autonomia tecnologica, muestra una preferencia en cuanto proveedores externos.
% - Regulación de transferencia de tecnología extranjera
%
% -Por tanto, la desagregación tecnológica es también un instrumento de política industrial que busca maximizar
% la participación propia en la ejecución de proyectos de inversión por medio del incremento del componente tecnológico
%  nacional en la producción de un bien o de un servicio. Para eso es preciso investigar, en detalle, los agregados humanos,
%   económicos y técnicos de cada uno de los proyectos de inversión.

% http://www.fundacionsadosky.org.ar/seminario-ciencia-y-tecnologia-en-el-pensamiento-de-jorge-sabato-oscar-varsavsky-y-amilcar-herrera/


\subsection*{Comparación con la posición de Rolando García}

Debido a la dificultad de encontrar textos específicos de Rolando García que hablen de políticas tecnológicas, trataré de abarcar su posición en el tema principalmente desde la charla que dio en el 2006 en la FCEyN.

En dicha charla, García menciona que:

\begin{itemize}
    \item ``Todo proceso profundo de transformación, en cualquier dominio, comienza con la apertura de nuevas vías de acción''.
    \item ``Las acciones emprendidas sólo tienen sentido si se comprende cuáles eran sus objetivos''.
    \item ``En el contexto particular de las universidades se discute mucho más sobre problemáticas puntuales que sobre el proyecto institucional en el que deberían de estar inscritas. Y es necesario subrayar que existe una gran diferencia entre un proyecto institucional y un programa de medidas propuestas o adoptadas en la coyuntura''.
\end{itemize}

Con lo cual me inclino a pensar que García entiende, como Bayer, que la conquista de la autonomía de los países en vías de desarrollo no se puede lograr de manera vegetativa, sin tomar acciones concretas al respecto. Es necesario buscar nuevas alternativas que permitan romper el ciclo de dependencia (lo que Bayer llama, explotar las potencialidades existentes).

Aún más, creo que concordarían en que no basta que ciertos actores particulares del sistema tomen algunas acciones concretas, sino que hacen falta políticas de estado que se enmarquen en un proyecto de conquista ( y luego mantenimiento) de la autonomía.

Con respecto a la autonomía universitaria, García dice:

\begin{itemize}
    \item ``Tener claro en qué consistía la autonomía universitaria y por qué era indispensable defenderla nos permitió negociar con diversas instituciones, incluso extranjeras, para obtener subsidios sin quedar subordinados a las decisiones que muchos intentaron imponer junto con los financiamientos''.
    \item ``Pero la autonomía no significaba, para nosotros, el aislamiento ante los problemas del país y del mundo. Casualmente, revisando en estos días viejos papeles, encontré un documento que vale la pena citar porque tiene que ver, precisamente, con las implicaciones políticas de nuestra posición. Mientras yo mantenía relación con la Fundación Ford y con otras organizaciones norteamericanas, en el año 1965 tuvo Lugar La invasión a Santo Domingo. En la Facultad de Ciencias Exactas y Naturales, expresamos nuestra condena con actos públicos''.
    % \item ``Cuando nos propusimos reconstruir la Universidad, en los años 50, enfrentamos una situación comparable en cierta medida. No era suficiente con modificar los planes de estudio, construir nuevos edificios y sustituir el personal docente y administrativo. Era necesario replantear La estructura de la institución, sus objetivos y su funcionamiento, para lo cual se requería mantener a ultranza la autonomía de la gestión. A partir de allí, \textbf{construir lo posible} significó encontrar los medios necesarios para poner en práctica la profunda reforma que nos planteamos como meta''
\end{itemize}

Creo firmemente que en este punto García estaría de acuerdo con la posición de Bayer, en que \textit{la política tecnológica deberá buscar satisfacer las necesidades sociales, buscando un aumento de la calidad en la vida de la población y no solo el aumento de la producción como meta principal}.

% http://www.pagina12.com.ar/diario/universidad/10-209827-2012-12-14.html
% Siempre entendió que las universidades y los centros de investigación debían estar al servicio de las necesidades populares y que, por lo mismo, la enseñanza y el aprendizaje debían apuntar al rigor y a la excelencia. Todo un jesuita este querido maestro y amigo.

% Charla 2006

% \item ``El haber conseguido Los subsidios de la Fundación Ford, insisto, para la compra de material científico y para otorgar becas externas, despertó ataques en todos los frentes. Por una parte, la derecha se preguntaba cómo era posible que fundaciones norteamericanas apoyaran a una facultad dominada por los izquierdistas y que privilegiara a Las ciencias duras por encima de Las humanidades. (Cabe recordar que la primera gestión que realicé personalmente en Washington con la Fundación, fue para conseguir fondos para La creación del Departamento de Sociología de la Facultad de Filosofía y Letras]. Por otra parte, la obtención de subsidios también despertó ataques por parte de la izquierda, que temía la penetración en el mundo universitario del imperialismo norteamericano a través de Los mismos.''
% \item ``Es necesario aclarar que los subsidios de la Fundación Ford fueron necesarios para el equipamiento de La Unrversidad y para otorgar becas al exterior, no para la construcción de la Ciudad Universitaria, que fue enteramente financiada por el Estado gracias a una partida especial gestionada directamente por Risieri Frondizi, rector de la UBA y hermano del Presidente de la Nación, Arturo Frondizi.''



\newpage
\section*{Pregunta 2}
\paragraph{Enunciado}
 Comparar la posición de Bayer con el planteo general del desarrollismo. ¿En qué se parecen y en qué se diferencian?

 \subsection{Comparación con la posición del desarrollismo}

 El planteo general del desarrollismo surge luego de la crisis mundial de 1929, con trabajos como la Tesis Prebisch-Singer que afirma que:
 \textit{los términos de intercambio en el comercio internacional, con un esquema centro industrial-periferia agrícola, tienden a caer a lo largo del tiempo,  y amplían la brecha entre países desarrollados y subdesarrollados.}


Otras posiciones centrales del desarrollismo se pueden encontrar en el artículo \textit{Commercial policy in the underdeveloped countries}, también de Prebisch:

% Youtube: Prebisch y los terminos de intercambio https://www.youtube.com/watch?v=sqUQQX1dTx8
 \begin{itemize}


    \item La industrialización es una parte ineludible en el proceso de cambio que acompaña un aumento gradual del ingreso per cápita.
    \item Además, no sirve que solo se tecnifiquen los sectores primarios, ya que en vez de que los trabajadores o los empresarios ganen más por ser más eficientes, los precios de los productos bajan, lo que beneficia a los países más ricos.
    \item Por lo tanto, la tecnificación de los sectores primarios puede ser favorable, pero debe estar siempre acompañada de un proceso más abarcativo y general de industrialización.
    \item En particular, Prebisch dice que la substitución de importaciones es la única forma de corregir los efectos de las disparidades en la elasticidad del comercio extranjero, de forma de lograr un adecuado crecimiento de los países periféricos.
    \item Sumado a lo anterior, la amplitud de los ciclos económicos en los países de la periferia es mayor que en los países del centro: cuando hay bonanza económica los precios de las materias primas aumentan porque aumenta la demanda, pero cuando hay contracciones estos caen aún más porque no hay mecanismos institucionales que frenen esa caída.
 \end{itemize}

Luego, en ese mismo artículo Prebisch menciona algunos puntos centrales que deberían seguir los países en vías de desarrollo para combatir las desventajas de su posición, y así romper la asimetría que genera el sistema capitalista dentro de los intercambios internacionales, lo cual, en mi opinión, permitiría lograr una menor dependencia de los países centrales y aumentar la autonomía:

 \begin{itemize}
     \item Mantener Estados activos, con políticas económicas que impulsen la industrialización.
     \item Utilizar medidas proteccionistas para disminuir los efectos del libre mercado en el crecimiento.
     \item Teniendo en cuenta que, a medida que la productividad y los salarios mejoran, las medidas proteccionistas deben ir bajando de forma gradual hasta que sean completamente eliminadas.
 \end{itemize}

 Las \textbf{similitudes} que encuentro con la posición de Bayer son:

 \begin{itemize}
     \item Un fuerte énfasis en la existencia de patrones de dependencia que mantienen a las naciones en vías de desarrollo en tal estado.
     \item Y que a su vez, hace falta un proyecto de Estado con políticas claras de conquista de la autonomía.
 \end{itemize}

 Y las \textbf{diferencias}:

 \begin{itemize}
     \item Después de la Segunda Guerra Mundial, muchos países (con un enfoque desarrollista) de América Latina deciden adoptar el modelo de Industrialización por Sustitución de Importaciones, que buscaba dejar de importar productos extranjeros y comenzar a consumir los producidos en el país de origen. Si bien este modelo trae consecuencias positivas como aumento del empleo, menor dependencia de los mercados extranjeros, mejora de los términos de intercambio, etc., también puede traer consecuencias negativas como la dependencia tecnológica cuando se importan nuevas tecnologías a ``paquete cerrado''. Esto último se resuelve generalmente tratando de realizar una desagregación de los paquetes tecnológicos, y creo que Bayer estaría de acuerdo en no hacer una importación \textit{ciega} de nuevas tecnologías.
     \item En las ideas centrales del desarrollismo, y en particular en el artículo que analicé de Prebisch, se habla mucho del crecimiento de un país como el aumento del ingreso per cápita, pero hoy en día sabemos que dicha mejora no necesariamente equivale a un desarrollo con equidad y bienestar social. Por lo tanto, encuentro que una de las criticas fuertes que podría hacerle Bayer al modelo desarrollista, es una falta de medidas que además de contemplar el crecimiento de los países en vías de desarrollo, contemplen la distribución de las riquezas generadas.
 \end{itemize}


\newpage
\bibliographystyle{plain}
\section{Referencias}
\begingroup
\renewcommand{\section}[2]{}
\bibliography{informe}
\endgroup

%\newpage
%\appendix
%\input{apendiceC}

\end{document}
