Debido a la dificultad de encontrar textos específicos de Rolando García que hablen de políticas tecnológicas, trataré de abarcar su posición en el tema principalmente desde la charla que dio en el 2006 en la FCEyN, UBA.

En dicha charla, García menciona que:

\begin{itemize}
    \item ``Todo proceso profundo de transformación, en cualquier dominio, comienza con la apertura de nuevas vías de acción''.
    \item ``Las acciones emprendidas sólo tienen sentido si se comprende cuáles eran sus objetivos''.
    \item ``En el contexto particular de las universidades se discute mucho más sobre problemáticas puntuales que sobre el proyecto institucional en el que deberían de estar inscritas. Y es necesario subrayar que existe una gran diferencia entre un proyecto institucional y un programa de medidas propuestas o adoptadas en la coyuntura''.
\end{itemize}

Estas ideas exhiben que García, al igual que Bayer, entiende que la conquista de la autonomía de los países en vías de desarrollo no se puede lograr de manera vegetativa, sin tomar acciones concretas al respecto, sino que es necesario buscar nuevas alternativas que permitan romper el ciclo de dependencia (lo que Bayer llama, explotar las potencialidades existentes).

Además, ambos muestran en su pensamiento que no basta que ciertos actores particulares del sistema tomen algunas acciones concretas, sino que hacen falta políticas de estado que se enmarquen en un proyecto de conquista (y luego mantenimiento) de la autonomía.

Con respecto a la autonomía universitaria, García dice:

\begin{itemize}
    \item ``Tener claro en qué consistía la autonomía universitaria y por qué era indispensable defenderla nos permitió negociar con diversas instituciones, incluso extranjeras, para obtener subsidios sin quedar subordinados a las decisiones que muchos intentaron imponer junto con los financiamientos''.
    \item ``Pero la autonomía no significaba, para nosotros, el aislamiento ante los problemas del país y del mundo. Casualmente, revisando en estos días viejos papeles, encontré un documento que vale la pena citar porque tiene que ver, precisamente, con las implicaciones políticas de nuestra posición. Mientras yo mantenía relación con la Fundación Ford y con otras organizaciones norteamericanas, en el año 1965 tuvo lugar la invasión a Santo Domingo. En la Facultad de Ciencias Exactas y Naturales, expresamos nuestra condena con actos públicos''.
    % \item ``Cuando nos propusimos reconstruir la Universidad, en los años 50, enfrentamos una situación comparable en cierta medida. No era suficiente con modificar los planes de estudio, construir nuevos edificios y sustituir el personal docente y administrativo. Era necesario replantear La estructura de la institución, sus objetivos y su funcionamiento, para lo cual se requería mantener a ultranza la autonomía de la gestión. A partir de allí, \textbf{construir lo posible} significó encontrar los medios necesarios para poner en práctica la profunda reforma que nos planteamos como meta''
\end{itemize}

Lo que indica que García comparte con Bayer la posición de que \textit{la política tecnológica deberá buscar satisfacer las necesidades sociales, buscando un aumento de la calidad en la vida de la población y no sólo el aumento de la producción como meta principal}.

% http://www.pagina12.com.ar/diario/universidad/10-209827-2012-12-14.html
% Siempre entendió que las universidades y los centros de investigación debían estar al servicio de las necesidades populares y que, por lo mismo, la enseñanza y el aprendizaje debían apuntar al rigor y a la excelencia. Todo un jesuita este querido maestro y amigo.

% Charla 2006

% \item ``El haber conseguido Los subsidios de la Fundación Ford, insisto, para la compra de material científico y para otorgar becas externas, despertó ataques en todos los frentes. Por una parte, la derecha se preguntaba cómo era posible que fundaciones norteamericanas apoyaran a una facultad dominada por los izquierdistas y que privilegiara a Las ciencias duras por encima de Las humanidades. (Cabe recordar que la primera gestión que realicé personalmente en Washington con la Fundación, fue para conseguir fondos para La creación del Departamento de Sociología de la Facultad de Filosofía y Letras]. Por otra parte, la obtención de subsidios también despertó ataques por parte de la izquierda, que temía la penetración en el mundo universitario del imperialismo norteamericano a través de Los mismos.''
% \item ``Es necesario aclarar que los subsidios de la Fundación Ford fueron necesarios para el equipamiento de La Unrversidad y para otorgar becas al exterior, no para la construcción de la Ciudad Universitaria, que fue enteramente financiada por el Estado gracias a una partida especial gestionada directamente por Risieri Frondizi, rector de la UBA y hermano del Presidente de la Nación, Arturo Frondizi.''
